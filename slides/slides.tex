%!TEX program = xelatex
\input{./env.tex}
\makeatletter
\def\input@path{{./fig/}}
\makeatother
\graphicspath{{./fig/}}

\title{Scala.js}
\author[Егор Горбунов]{
	Студент: Егор Горбунов
}
\institute{CПбАУ}
% \date{20 мая 2016 г.}

\begin{document}
\maketitle
\begin{frame}{Слайды и примеры}
\begin{center}
	\textbf{\texttt{\href{https://github.com/egorbunov/scala-js-lecture}{https://github.com/egorbunov/scala-js-lecture}}}
\end{center}
\end{frame}

\begin{frame}{JavaScript}

\begin{itemize}
	\item Исторически сложившийся стандарт для оживления ваших веб-страниц
	\item Браузер предоставляет \emph{объектную модель документа} (DOM), которую можно изменять (из JavaScript)
	\item AJAX (Asynchronous Javascript and XML)
\end{itemize}

\end{frame}

\begin{frame}[fragile]
\frametitle{Проблемы JavaScript}
\begin{itemize}
	\item \sout{Я не знаю JavaScript}
	\item Динамический
	\item Нетипизированный
	\item Непредсказуемый
\begin{lstlisting}[escapeinside={(*}{*)}]
['10','10','10','10','10'].map(parseInt) (*$\Longrightarrow$*) [10, NaN, 2, 3, 4]
\end{lstlisting}
\end{itemize}
\end{frame}

\begin{frame}{Что такое Scala.js?}
\begin{itemize}
	\item Компилятор из \alert{Scala} в \alert{JavaScript}
	\item Получаем альтернативу JavaScript для \emph{frontend} разработки
		\begin{itemize}
			\item[-] Статическая типизация $\rightarrow$ безопаснее
			\item[-] Один язык для серверной и клиентский частей
		\end{itemize}
	\item Совместим с JavaScript (т.е. можно вызывать из JavaScript из Scala.js и наоборот)
\end{itemize}
\end{frame}

\begin{frame}{Процесс компиляции в Scala.js}
	\begin{itemize}
		\item[] \alert{\texttt{sbt> compile}}
		\begin{center}
		\begin{small}
			\texttt{*.scala}\\ 
			$\Downarrow$ \\ 
			\texttt{*.sjsir} $+$ \texttt{*.class}
		\end{small}
		\end{center}
		\item[] \alert{\texttt{sbt> fastOptJS}}
		\begin{center}
		\begin{small}
			\texttt{\{*.sjsir\} + \{*.class\}}  \\ 
            $\Downarrow$ \\
            граф достижимости символов \\
            $\Downarrow$ \\
            \texttt{fastOpt.js}
        \end{small}
		\end{center}
		\item[] \alert{\texttt{sbt> fullOptJS}}
		\begin{center}
		\begin{small}
			\texttt{fastOpt.js} $\Rightarrow$ \texttt{\href{https://github.com/google/closure-compiler}{Google Closure Compiler}} $\Rightarrow$ \texttt{fastOpt.js}
        \end{small}
		\end{center}
	\end{itemize}
\end{frame}

\begin{frame}{Пример 1. Scala.js и только}
\begin{itemize}
	\item Используем \alert{Intellij Idea} $+$ \alert{Scala Plugin} $+$ \alert{SBT Plugin}
	\item Scala $2.11.8$
	\item Scala.js $0.6.13$
\end{itemize}
\begin{itemize}
	\item \alert{\texttt{sbt> run}} --- исполнить код Java Script интерпретатором
	\begin{itemize}
		\item по умолчанию используется \emph{Node.js}
	\end{itemize}
\end{itemize}
\end{frame}

\begin{frame}{Пример 2. + Workbench + Scalatags}
\end{frame}

\begin{frame}{Пример 3. + Autowire + Akka}
\end{frame}

\end{document}
